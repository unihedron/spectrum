\documentclass{article}
  \usepackage{graphicx}
\usepackage{amsfonts}
\usepackage{amsmath}
\usepackage{pstricks}
\usepackage{auto-pst-pdf}
\usepackage{pgfplots}
\pgfplotsset{%
	compat=1.14,
	axis line style={draw=none},
	tick style={draw=none},
	xticklabels=\empty,
	yticklabels=\empty,
	domain=0:1000,
	samples=1e3,
}
\begin{document}

\begin{pspicture}(5,5)
\pspolygon[linejoin = 1](1,1)(5,1)(1,4)
\pscircle[linestyle=dotted](3,2.5){2.5}
\pscircle[fillstyle=solid,fillcolor=lightgray](2,2){1}
\psline{o-}(.08,.6)(.5,.6)\uput{2pt}[0](.5,.6){White Hot}
\end{pspicture}


\newcommand{\nA}{10}% Start frequency.
\newcommand{\nB}{110}% End frequency - start frequency.
\newcommand{\nC}{.1e3}% Number of samples.
\newcommand{\nM}{0.5}% Multiplying factor for amplitude.

\pgfplotsset{
	colormap={bluered}{
		rgb255(0cm)=(128,0,0); rgb255(1cm)=(255,0,0); rgb255(2cm)=(255,255,0);
		rgb255(3cm)=(100,255,0);rgb255(4cm)=(0,255,255); rgb255(5cm)=(0,0,180)},
	domain=1:\nC
}

\begin{tikzpicture}[rotate=90]
	\begin{axis}[width=11in,height=1in]
		% http://www.fon.hum.uva.nl/praat/manual/Script_for_creating_a_frequency_sweep.html
		\addplot [black,line width=3pt,line join=round, line cap=round]             {\nM * sin(\nA * x + \nB * (x^2) / \nC)};
		\addplot [mesh,point meta=x,line width=2pt,line join=round, line cap=round] {\nM * sin(\nA * x + \nB * (x^2) / \nC)};
	\end{axis}
\end{tikzpicture}

\end{document} 
